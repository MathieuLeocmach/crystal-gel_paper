\documentclass[11pt]{article}
\usepackage{graphicx}
\usepackage{amssymb}
\usepackage{epstopdf}
\usepackage{color}
\usepackage{setspace}

\begin{document}

\doublespacing



\doublespacing

\noindent
Dear Editor,
\vskip 0.3cm
we would like to thank you for your comments to our manuscript ``A novel route to the spontaneous formation of porous crystals via viscoelastic phase separation'' (NM15113574),
and for providing us with three Referee reports.
We were happy to see that all Referees appreciated our work, and at least two of them suggested
resubmitting it after some technical clarifications. Referee 1 describes our results as ``very nice and beautiful'' but does not recommend our work
based on the claim that it is not novel. He refers in particular to a series of works by the Edinburgh group in which the kinetics of phase
transitions is classified according to the free energy landscape. These work are by themselves very important and relevant, but contrary
to the Referee claim, do not contain a description or results on the crystallization channel we describe in our work.
Both Referees 2 and 3 have appreciated this fact.

To address the technical questions of Referee 2 and 3 we have conducted novel experiments, that clearly demonstrate the reproducibility
of our work. In particular, we believe that the concerns of Referee 3 are simply due to the fact that the two movies in Supplementary Informations
are shot with different framerates. This gives the impression of abundance of gas phase, that probably led the Referee into thinking that the Ostwald
ripening process is dominating. We have now made this issue more clear, and are confident that it will answer the Referee's concern fully.
We also broadened our analysis, to include a direct study of the trajectory history to determine the different crystallization
channels. As a result, new sections have been added to the Supplementary Informations, providing for a in-depth demonstration of the generality of our results.

We hope that you will find that we have fully addressed the Referee questions, and
that you will find the new manuscript to be suitable for publication in \emph{Nature Materials}.




\vskip 0.8cm

\noindent
Yours Sincerely,
\vskip 0.3cm
\indent Hideyo Tsurusawa\\
\indent John Russo\\
\indent Mathieu Leocmach\\
\indent Hajime Tanaka

\clearpage

\noindent
\begin{Large}
{\bf Response to Referee \#1}
\end{Large}

\vspace{1em}

\singlespacing

{\bf
This manuscript describes experimental work on a colloid-polymer mixture that undergoes viscoelastic phase separation thereby forming a spinodal structure, which is interrupted by crystallization. The authors argue that this mechanism can be used to form nano-porous crystals, which has potential applications. However, this phenomenon has been shown in extensive previous studies, see Ref. [7-10], and is therefore not novel. Another key result of the paper is that the authors demonstrate that the crystalline branches of the spinodal network can also grow by the attachment of "gas"-like particles onto the surfaces. Even this message is not completely new as the Edinburgh group (Poon, Evans, Renth, Cates) has made extensive studies on the precise kinetic pathways of the phase transformations in colloid-polymer mixtures and explained this by means of free-energy landscapes. An extensive body of work that the authors apparently missed. Although the experimental study and analysis is very nice and beautiful, the 
referee hesitate to recommend the manuscript for publication in Nature Materials due to a lack of novelty (in terms of material and fundamental insight) of the present work.
}


\bigskip
\doublespacing

First We would like to thank the Referee for her/his careful reading of our manuscript, for her/his comments. 

The main concern of the reviewer is the novelty of our work in the light of Ref. [7-10] and the previous pioneering works on the related issue by the Edinburgh group. In the original manuscript we only cited a review by Poon, but, as the Referee correctly points out, we should have
also included papers by Poon, Evans, and Renth and the Edinburgh group on phase separation into coexisting colloidal gas, liquid, and crystal regions, and its link to the free energy landscape.

However, we have very carefully read these previous references and confirmed that our new mechanism was not mentioned before. This is supported by the kind comments on this point made by Reviewer 2. The references predict the kinetic pathway from the free energy landscape, and the discussion
is based on macroscopic experiments under the influence of gravity. These experiments do not contain information on the kinetic pathway at a microscopic level.
In the paper by Renth et al., the closest example to our crystalizzation mechanism is “Sample 5” belonging to the kinetic regime called “G”. In this case, the process is initiated by spinodal decomposition. They inferred that any crystallites nucleated at the early stage must have gas coatings, on the basis of the non-existence of a liquid-crystal common tangent. The subsequent theoretical paper by Evans et al. also stated that spinodal decomposition takes place in the region G before crystallization takes place. Then they cited the simulation work by Soga et al. (Ref. 7) and 
stated that spinodal-like instability leads to a texture of high and low density amorphous regions and then crystals subsequently formed in one of these regions. This statement means that crystallites are formed in either the gas or the liquid phase. Overall, the process is different from what we found in our paper, which is a pathway involving all the three phases in a non-trivial manner. Unveiling the exact kinetic pathway of crystal-gel formation cannot
be done exclusively from the free energy-landscape, and requires a particle-level study of the process, which is achieved for the first time by our experiments.

We hope that the above explanations would convince the reviewer that our paper reports a novel kinetic pathway to porous crystals, which has not been discussed before. 

As explained in our replies to the other reviewers, we have revised our manuscript incorporating all the reviewers’ comments including the above concern on the novelty of our work. We hope that the Reviewer would find that our revised manuscript is now suitable for publication in Nature Materials. 


\clearpage

\noindent
\begin{Large}
{\bf Response to Referee \#2}
\end{Large}


\vspace{1em}

\singlespacing

{\bf
The authors report on a very interesting and new result: systems with short-range attractions that undergo freezing from a gas phase can form a solid with short-rang crystalline order but long-range, gel-like amorphous (porous) structure. The experiments support the conclusions quite well and the results are explained in a convincing way using the idea of metastable states. While previous researchers have shown that a short-range attraction can lead to gels (which is well known) or an intermediate fluid state on the way to crystal formation, this is the first report of early-time crystal formation that crosses over to spinodal-like gel structures at later times. I note in passing that this is different from early-time crystallization followed by random aggregation of these crystallites, which would be expected to yield fractal aggregates of crystalline domains, and not the porous or bicontinuous structures reported here. Therefore, I think the results are quite new. I think the results are relevent to the 
materials community and I recommend publication after a few comments are addressed.
}


\bigskip
\doublespacing

First We would like to thank the Referee for her/his careful reading of our manuscript, for providing us with valuable comments. We are very glad to see such kind positive comments on our work.


\vspace{1em}

\singlespacing

{\bf
The authors propose that this method might work in a variety of other materials and be useful as a generalizable, rational method to make materials that are crystalline yet porous. I agree with them about the potential advantages of such materials. However, I am less optimistic than they are about the possibility of this mechanism in small-molecule systems. For example, the authors propose that this could work for metals - this is mentioned in the abstract and in the conclusion, on page 9. Success in this area requires meeting the 4 conditions listed on pages 8-9 but this may not be possible with nobel metals (especially with regard to iii). Choosing a different example might make this point more credible and the article more valuable. Alternatively, if the authors have good arguments for why this could work for noble metals, then these could be a good addition to the article.
}



\bigskip
\doublespacing

Below we will answer the comments one by one. 


\vspace{1em}

\singlespacing

{\bf
Overall the methods are appropriate to the conclusions. The article is quite clear other than the points listed below.
}


\bigskip
\doublespacing

Below we will answer the comments one by one. 


\vspace{1em}

\singlespacing

{\bf
A few technical points:

1) Concerning references: Page 8, line 7: "kinetic path via the gas phase (liquid $\rightarrow$ gas $\rightarrow$ crystal) has been completely overlooked so far" - does this statement refer to experiments? I believe that the literature has covered a wide range of theory on this topic, for example in articles by Evans et al or Renth et al with WCK Poon (2 articles in PRE, 2001). One of those articles also reports experiments, which might be worth citing here (though I do not think those experiments are the same as these findings).
}


\bigskip
\doublespacing
We should have cited these pioneering works on the pathway whereby a system phase separate into macroscopically coexisting colloidal gas, liquid, and crystal regions and its link to the free energy landscape. We have carefully read these papers. In the paper by Renth et al., the closed example to ours is their “Sample 5” belonging to the kinetic regime called “G”. In this case, the process is initiated by spinodal decomposition. They inferred that any crystallites nucleated at the early stage must have gas coatings, on the basis of the non-existence of a liquid-crystal common tangent. But this does not tell anything on the kinetic pathway on a microscopic level. We stress that in these previous studies there was no assess to particle-level information and the discussion is based on macroscopic experiments under an influence of gravity. The subsequent theoretical paper by Evans et al. also stated that spinodal decomposition takes place in the region G before crystallization takes place. Then they cited the 
simulation work by Soga et al. (Ref. 7) and stated that spinodal-like instability leads to a texture of high and low density amorphous regions and then crystals subsequently formed in one of these regions. This statement means that crystallites are formed in either the gas or the liquid phase. Thus, this is clearly different from what we found in our paper as a new pathway involving all the three phases in a non-trivial manner. 
Nonetheless, our statement "kinetic path via the gas phase (liquid $\rightarrow$ gas $\rightarrow$ crystal) has been completely overlooked so far" was obviously too strong. We have replaced “completely” by “largely” and have cited these pioneering works, which considered the role of the complex free energy landscape for the first time. 



\vspace{1em}

\singlespacing

{\bf
2) Page 5, second paragraph: the text states that the vol frac of 68\% corresponds to the vol frac of the stable crystal phase. This contradicts the phase diagram shown in Fig 1A, which has the coexisting crystal vol frac at about 0.72 or so - far over toward the right-hand axis. This should be quite distinct from 0.68. There may be good reasons for the difference between the measured 0.68 and the expected equilibrium value, but this should be discussed.
}




\bigskip
\doublespacing

We were not clear enough on this point. In the experimental representation of the phase diagram (vertical axis is $c_p$), the tie lines are not horizontal. We have now drawn the slanted gas-crystal tie line in the figure as a guide. We hope that this would make the situation much clearer.
\textcolor{red}{Mathieu could you check this point?}


\vspace{1em}

\singlespacing

{\bf
3) How were crystalline particles defined? Page 6 refers to measurements of crystallite size and number, and the definitions should be given. Also does "amount of crystals" (p6) refer to number or volume?



4) How is the local volume fraction defined? Is it the particle volume divided by the volume of the Voronoi cell?
}


\bigskip
\doublespacing

We apologize that these technical details were not sufficiently explained in the previous version of the manuscript.
We have now added a section to the Supplementary Informations detailing the exact definition of crystalline particles
(and the other states too). In brief, crystallinity is measured by means of bond orientational analysis, with order
parameters designed to detect all relevant crystalline phases for colloidal/polymer mixtures (fcc, hcp and bcc).
The amount of crystals always refers to number and not volume.

The Referee is correct regarding the volume fraction. It is defined by calculating the Voronoi diagram of each configuration
and computing the fraction of volume that each colloid occupies in its cell. We have also added this information to
the manuscript.


\vspace{1em}

\singlespacing

{\bf
5) On pages 4 and 5, the authors refer to "random gelation universality class" and the associated exponent D=2.53. To what do the authors refer? I would have thought that the appropriate comparison would be a particle-aggregation model such as DLCA or RLCA, which have D=1.75 and 2.0 or greater. Please explain and justify this comparison.
}




\bigskip
\doublespacing

By "random gelation universality class" we mean three-dimensional percolation universality class (which applies for example to randomly branched polymers).
It is known that three-dimensional random percolation clusters and diffusion-limited aggregation clusters have both about the same fractal dimension of about 2.5, for which we have cited the review article by Herrmann. We agree that DLCA and RLCA provide smaller fractal dimensions; however, DLCA is usually observed at much lower volume fraction and with stronger bonding energy (after bonding there is little diffusion, which leads to the formation of really fractal objects) than our case. For our starting volume fraction of about $\phi\approx 0.30$ and polymer concentrations, we have found that the fractal dimension
to be more consistent with random percolation. We hope that this clarifies the situation.


\vspace{1em}

\singlespacing

{\bf
6) A small technical point: 'sublimation' is used to refer to the formation of a crystal from vapor at the bottom of page 7 and on page 8 (8 lines from the bottom). Sublimation is a transition from solid to vapor, not the other way.
}



\bigskip
\doublespacing

We thank the Reviewer for pointing out this mistake. We have corrected it as suggested. 


We hope that the Reviewer would find that the revised manuscript is now suitable for publication in Nature Materials. 


\clearpage

\noindent
\begin{Large}
{\bf Response to Referee \#3}
\end{Large}


We would like to thank the Referee for her/his careful reading of our
manuscript, and for her/his comments.

\vspace{1em}

\singlespacing

{\bf
Tsurusawa et al. report measurements of crystal gels. Crystal gels are described as crystalline networks that form by a two-step mechanism. First, a metastable spinodal crystallization boundary induces liquid-gas phase separation; crystals then nucleate within the liquid phase. Second, the crystals grow by a process in which liquid droplets evaporate into the gas phase, then desublime to the crystal. This process is viewed as akin to the Bergeron process of ice crystal growth in clouds. I have the following specific comments:
}



\bigskip
\doublespacing


First we would like to thank the Referee for her/his careful reading of our manuscript, for providing us with valuable comments. Below, we will answer the comments one by one. 


\vspace{1em}

\singlespacing

{\bf
1) The evidence for this crystal gel mechanism appears to derive from measurements at a single state point. For this sample, what happens at the transition between the two SI movies, at about 500 Brownian times? At early times, I see very few free particles, and no crystallization. A few frames into the longer movie, I see many more free particles, with exchange between the dense and dilute regions, mostly from one crystallite to the other. What, within the scenario proposed, would lead to such an abrupt increase in the abundance of free particles at ~ 500 Brownian times? It is as if the system itself changed its state point and shifted over a boundary that then allowed the crystallization to accelerate. (The increase in free particles occurs more at ~ 500 Brownian times occurs more rapidly than I think would be expected based on the kinetics of spinodal decomposition.) I think that additional observation of the phenomena of interest at different state points and/or evidence for replication at the indicated 
state points are needed to rule out idiosyncratic (time-dependent) physical chemical explanations for the abrupt increase in gas phase colloids.
}



\bigskip
\doublespacing


We think that we have not been clear enough about the explanations of the Supplementary movies. Movie 1 and Movie 2 were made with different frame rates and there was an interval of 90 s between them. This is because the dynamics is rather fast in the initial stage driven by spinodal decomposition. But, in the late stage, the dynamics slows down and thus we reduced the frame rate of 3D scan to cover the long-time period. We are afraid that the Reviewer’s impression of a sudden increase in free gas particles and a possible discontinuity between the two movies originatse from this difference in the frame rate.

To address the concern about the reproducibility of our kinetic pathway, we decided to make additional independent experiments
by using a sample with similar composition, which is inserted into an ordinary glass capillary tube. We thus confirm the reproducibility of this type of crystallization pathway. We have added these additional experiments in the Supplementary Information. We would like to thank the Referee for stimulating
this additional investigation.


\vspace{1em}

\singlespacing

{\bf
2) From the SI movie, I don't recognize the final reported structure to be a crystal gel. These final structures seem instead to be crystallites of finite size at equilibrium. The crystallites diffusive minimally because they are jammed against their neighbors. The results seem similar to Anderson and Lekkerkerker's (ref 20) explanation of Ostwald ripening. The changes in crystal size at late times, when the liquid to gas to crystal mechanism is proposed, appear that they could be equally well described as a process in which larger crystallites grow at the expense of smaller ones, through the gas phase.

}



\bigskip
\doublespacing

The impression of the reviewer that the final structures are made of crystallites of finite size may come from the fact that the percolated network structure cannot be seen by 2D sectioning. The Movies show only a thin slice of the system which corresponds to the particle that are in focus.
To show the connectivity of the network clearly, we have also made Supplementary Movies of a thin slice with a finite thickness. We hope that this will clarify the situation and convince the reviewer of the fact that our sample indeed forms a porous network of crystals. 


\vspace{1em}

\singlespacing

{\bf
3) The data quality of Figure 1c, Figure 1d appear insufficient to classify the cp = 0.38 mg/mg state point as different from the others. The power law scaling in Figure 1c is generated over at most half a decade and the difference between 68\% and 64\% in Figure 1d at 12 nearest neighbors seems marginally resolvable given what the errors might be. The authors should add replications, errors, and significance testing if they wish to distinguish the dimensionality (Fig 1c) and maximum volume fraction (Fig 1d) of the 0.38 mg/ml state point from the others.
}



\bigskip
\doublespacing

We can partly agree that the results of Fig. 1c/d would not be enough to differentiate the  cp = 0.38 mg/g state from the others
state points, if considered alone. The spirit of these figures is to show all the information we can get before we utilize the bond orientational
analysis to detect crystalline particles. Bond orientational analysis, as reported in Fig.2/3/4, confirms
that indeed the state point cp = 0.38 mg/g is affected by crystallization. In particular, the dimensionality obtained
with the structure factors, will be examined in more detail in Fig. 3b, where single-particle statistics allows us to
get rid of all the noise.
Regarding the volume fractions of Fig. 1d, we have performed the error analysis suggested by the Referee, by looking at the
standard deviations of the volumes of particles with different neighbours. The results are reported in the updated Fig. 1d,
where we indeed confirm that our results are statistically sound.



\vspace{1em}

\singlespacing

{\bf
4) In panel 4b, about ~ 25\% of the particles are identified as in the surface state; however, transition probabilities are not recorded for this state in Figure 4c. From the SI movies, it seems that most of the kinetics are between surface and gas states at long times (as in Ostwald ripening). I did not find explanation in the paper about how these transition probabilities are defined and measured, and how the surface particles are excluded from the analysis. Also, do the kinetics measured satisfy detailed balance at long times? These data should include crystal-gas and crystal-liquid transition probabilities so that the author's mechanism of liquid to gas to crystal can be evaluated against the alternative that the sublimination/desublimination transitions are from crystal to gas to crystal.
}

In the previous version of the manuscript, Fig. 4c included surface particles together with crystalline particles just to improve
the readability of the graph, but we agree with the Referee that by doing so we didn't convincingly show that the Bergeron channel (liquid-to-gas)
is more prominent than the Ostwald ripening channel (crystal-to-gas). 
To include the full analysis, as requested by the Referee,
we have added a section to Supplementary Information, with a new figure (Fig. S3) containing four panels that detail the transition
probabilities between all states. In particular we note that, while being an active channel, the Ostwald ripening pathway is limited by
the small conversion probability of surface particles into the gas phase.






\vspace{1em}

\singlespacing

{\bf
5) Looking at Figure 4b and 4c together: although the gas crystal transition probability is large, the abundance of gas particles is never very significant. In the key window between 2 and 6 thousand Brownian times, when the crystal forms, I do not see how Figure 4c discriminates between a liquid to gas to crystal scenario and a direct liquid to gas scenario. (I agree that at long times there are abundant gas to crystal and crystal to gas transitions.) But, at four thousand Brownian times, the low gas concentration relative to liquid concentration and the equivalence of the transition probabilities suggests to me that the direct liquid to crystal route is what actually builds this crystal. The authors describe this as an initial nucleation process; however, from these figures it appears that ~ 80\% of the crystallization actually occurs in this window of initial nucleation, and this process is the typical one (reviewed for example in ref 20).
The evidence for the formation of crystals in this paper is convincing (Figure 2d and Figure 4b). However, because of the weaknesses in the evidence presented in Figure 1c,d and Figure 4b,c, as well as from watching the supplemental movies, I do not think that the authors have adequately proven that they have observed a new crystallization pathway. First, at about ~ 4000 Brownian times, when the crystals form, it seems like a liquid to crystal transition pathway explains the data as well as the proposed liquid to gas to crystal pathway, and the bulk of the crystallization occurs at this point (rather than later through the proposed desublimation pathway). Second, at longer times, these solutions seem to be understandable as diffusing crystallites - the connection to gelation seems unnecessary to understand the physics. The late state dynamics instead seems like Ostwald ripening, in which smaller crystallites transition to larger one through sublimination; the liquid does not appear to play an intermediate 
role here and there is little evidence for the liquid to gas to crystal route that is the paper's primary message.
}



\bigskip
\doublespacing

As mentioned before, we are afraid that the above interpretation basically originates from some misunderstandings, which is caused by our poor explanations on the different frame rates of Supplementary Movies and the lack of more 3D representations of our results. We hope that the above explanations and new Movies would convince the reviewer of the formation of a percolated network structure composed of crystals (not isolated crystals diffusing around) and the novel kinetic pathway leading to porous crystalline structures.
Moreover, to address directly the Referee's concerns we have added a new analysis, whose results are reported in the new Fig. 4c.
The analysis is based on a classification of the trajectory of
each particle. By considering the different state transformation that a particle undergoes, we are able to classify its trajectory as belonging to one of three different crystallization channels: direct crystallization, Bergeron and Ostwald. Direct crystallization refers to trajectories in
which gas or liquid particles transition to the crystal state without liquid evaporation or crystal de-sublimation in between the two states.
The Bergeron process refers to trajectories in which liquid particles transition to the gas state before attaching to the nucleus. The Ostwald
process refers to trajectories in which a crystal particle transitions to the gas state before crystallizing again.
The results show that, over the whole process, 54\% of the trajectories crystallize directly, 34\% take the Bergeron route, and
11\% instead can be attributed to Ostwald ripening. We hope that this analysis can convince the Referee that the Bergeron process is a
fundamental new channel of crystallization, that plays an important role in the growth phase of crystallization.


\vspace{1em}

\singlespacing

{\bf

Minor points:

A) I think the previous literature about Bernal spirals and other one-dimensional structures in gels (e.g. PRL 94 208301 2005) is relevant to the present study.
}




\bigskip
\doublespacing

We thank the reviewer for notifying this important reference. We have cited it in our revised manuscript. 


\vspace{1em}

\singlespacing

{\bf
B) The SI movies would be more effective if they were merged and a time stamp added; that would allow the reader to better compare them to Figure 4.
}


\bigskip
\doublespacing
We have followed the kind suggestion of the reviewer. 


\vspace{1em}

\singlespacing

{\bf
C) What time point or condition is Figure 3b at? Please add this information to distinguish these data from Figure 1b.
}



\bigskip
\doublespacing

We have followed the kind suggestion of the reviewer. 



We hope that the Reviewer would find that the revised manuscript is now suitable for publication in Nature Materials. 


\end{document}
