\documentclass[11pt,a4paper]{article}
\usepackage{graphicx}
\usepackage{amssymb}
\usepackage{epstopdf}
\usepackage{xcolor}
\usepackage{setspace}

\usepackage[top=35mm,bottom=35mm,left=30mm,right=30mm]{geometry}

\newenvironment{referee}%
{\bigskip\singlespacing\bf}%
{\par\bigskip}


\begin{document}

%\doublespacing



%\doublespacing

\noindent
{\bf Dear Pep,}
\vskip 0.3cm
We would like to thank you for your providing us with two Referee reports for our manuscript (NM15113574). We understand your decision to reject the manuscript, but, after careful consideration of the Referee reports, we couldn't avoid thinking that they are mostly supportive of our work, and the only Referee (Referee \#1) that is somewhat against publication is basing his argument on a probable misunderstanding of the true novelty of the first part of our work.

We decided to write to you again to clarify that the Referee might have missed some very important points. In general, both Referees seem to have appreciated the improvements
we made after the first round of reviews: Reviewer \#1 calls the work \emph{very beautiful and interesting}, while Referee \#3 goes into the details of explaining how we actually satisfactorily clarified his previous doubts, with just a couple of minor points left. Reviewer \#2 was very positive during the first round, but unfortunately didn't submit a second report.
Overall we were able to find only two criticism in the reports.
\begin{itemize}
 \item Referee \#1 laments that the first part of the paper describes spinodal decomposition without showing anything new from previous studies (but he admits the novelty of the study of the different crystallisation channels). But this is not the case, as the crystallisation of the liquid network requires an additional step, called stress-driven aging, that was never observed experimentally before.
In our new version of the manuscript we spent considerable time developing a new analysis, that shows how spinodal decomposition proceeds differently depending on the polymer concentration, and that the aging of the network due to mechanical stress is a necessary condition for the subsequent crystallization process. We think that this completely invalidates the point of the Referee, because in addition to different crystallization channels, our study unveils a new link between the aging of the liquid network and the subsequent crystallization process.
 \item Referee \#3 suggests that we better acknowledge that the crystal-gel is found only in a quite specific region of the phase diagram, and this also became the main motivation for the rejection of our manuscript. We also fundamentally disagree with this idea. It's true that crystal-gel states don't appear everywhere in the phase diagram, but this doesn't mean that they are hard to obtain. Rather the opposite, there is a reproducible region around the critical point were these states can be consistently obtained, and we were able to reproduce these findings both in our new cell design, and in ordinary capillaries, showing that the region of crystal-gel formation is well within the experimental range. Moreover the extent of this region can be widened by changing the location of the critical point, which in our system is controlled by the length of the non-adsorbing polymer: moving the critical point to lower polymer concentrations opens the window where bonds can rearrange before the intervening glass transition. So we argue that the region of formation of crystal-gel porous structures is not only easily accessible, but also controllable.
\end{itemize}

We hope that you'll see that both criticisms are not indicative of any weaknesses in our work. We also took advantage of the new suggestions to improve the manuscript even further. In particular, concerning the point raised by Referee \#1 we have introduced a new type of analysis, which looks at the bond-breaking probability as a function of time, during the whole process of spinodal decomposition. Our new results show that, at low polymer-concentrations, the network undergoes a structural reorganisation, which results in more compact structures, that are a necessary condition for the appearance of the first crystalline nuclei. To our knowledge this is the first experimental demonstration of this process: it is not captured by computer simulations, as they generally neglect hydrodynamic interactions which play a fundamental role in this kinetic process, and no previous experiment has had the capability of tracking particles during the early stages of spinodal decomposition.


We are rather positive that the new version of the manuscript would put the Referees in a position to better understand the full extent of our results, and
so we hope that you might give it another chance. If that is is not the case, we are still grateful for the opportunity that the comments gave us for improving the manuscript in a meaningful way.

\vskip 0.8cm

\noindent
{\bf Yours Sincerely,
\vskip 0.3cm
\indent Hideyo Tsurusawa\\
\indent John Russo\\
\indent Mathieu Leocmach\\
\indent Hajime Tanaka}

\clearpage

\noindent
\begin{Large}
{\bf Response to Referee \#1}
\end{Large}

\vskip 0.5cm

We would like to thank the Referee for her/his comments. We address these points in the following response.

\begin{referee}
The authors have improved their manuscript according to the comments of all referees, which has increased the readability of the paper. The novelty of the paper is also much better explained, which is in fact described not earlier than on page 7 of the manuscript.
The first 7 pages of the paper describes one of the arrest mechanisms of spinodal decomposition due to crystallization, which was already observed in experiments and simulations.
\end{referee}

We thank the Referee for her/his appreciation of the improvements made during our last revision of the manuscript. The Referee comments on the first 7 pages being not original, and simply describing the process of spinodal decomposition as previously observed in experiments and simulations. But we believe that our study offers novel insights also in the description of the early stages of the phase separation process. This is a very important point, as there is a clear link between the early-stage kinetics of the liquid network, and the late-stage crystallization process. In order to highlight this connection, in this new version of the manuscript we introduce a new type of analysis. In summary, we show that our work represents the first experimental demonstration of a process called \emph{stress-driven aging}. Most past simulation works have neglected, or not fully taken into account, the role of hydrodynamic interactions (HI) in the phase separation process. As was shows for example in Ref.~\cite{furukawa2010key}, the phase separation process without HI produces rather compact structures, which would then easily crystallize to form compact nuclei. This is not what happens if one take HI into consideration. Due to the incompressibility condition, the liquid network has a lower-connectivity, and the viscoelastic phase separation results in the accumulation of mechanical stress on these liquid bonds. To things can happen depending on the polymer concentration. At high-polymer concentration, the bonds are not-easily broken, and the network which is first formed cannot undergo structural reorganisation. This inevitably leads to gelation. On the other side, if the polymer concentration is low enough, the liquid network can reorganise with a rather sharp transition to a more compact network. This condition is necessary for the appearance of the first crystalline nuclei. In our new manuscript, we are able to follow the kinetics of these processes by looking at the time evolution of the bond-breaking probability. The different network states are in fact characterised by different bond-breaking probabilities and by different connectives. In Fig.~2a,b we show these quantities for high-polymer concentration samples, where spinodal decomposition is the only process that occurs. At low-polymer concentrations (Fig.~2c,d) instead, the network undergoes a sharp transition to a state with lower bond-breaking probability and higher connectivity. This transition is the stress-driven aging process, and increases the connectivity of the network. Despite being theoretically predicted, Ref.~\cite{tanaka2000,tanaka2007spontaneous}, this transition was not directly observed before.
We agree that the previous version of the manuscript was lacking on this part, so the Referee comments inspired us to develop the new analysis presented in Fig.2.


\begin{referee}
What is new is that crystallization not only takes place in the spinodal liquid branches, but can also occur via the Bergeron process in which the liquid evaporates, and the gas phase crystallizes again or via Ostwald ripening, i.e., melting of crystallites into the gas phase and recrystallization. Both processes make sense and can be explained that the crystallites should be surrounded by the gas phase due to the non-existence of a liquid-crystal coexistence as already speculated by Renth et al. It would be worthwhile to mention this more explicitly in the paper. In addition, the authors mention on page 3, that for confocal microscopy studies on colloidal gelation, the colloid-polymer mixture have so far been used. There are also confocal microscopy experiments performed on oppositely charged colloids,
which is worthwhile mentioning as well, Sanz et al., J. Phys. condens. Matt. 20, 494247 (2008), Sanz et al., J. Phys. Chem. B 112, 10861 (2008), where the effect of range and attraction strength have been studied on the different arrest mechanisms of the gels, including the crystallization of spinodal structures. Although the work is very beautiful and interesting, the referee has still strong doubts if the work provides sufficient novel insights to warrant publication in Nature Materials. First of all, the title suggests that a completely new route is found to form porous crystals. Crystallisation of spinodal networks has already been found in previous works and so the present work doesn't provide a new route, only another crystallization mechanism when the crystals get larger than the thickness of the branches. Secondly, the abstract suggests a new confocal microscopy protocol to study phase separation, which is in fact based on a salt injection that changes the interactions in
situ. Finally, the last sentence of the abstract is also misleading as this new route is not a single-step pathway to form nanoporous crystals, it involves spinodal decomposition, crystallization of the branches, and subsequently evaporation of the liquid and crystallization or melting of crystalline parts and recrystallization.
\end{referee}

In the new version of the manuscript we have tried to incorporate all the Referee's previous suggestions. We have added the recommended references to the new version of the manuscript~\textcolor{red}{I remember the sensei saying that he had included some of the References; do we need any else?}. We simplified the title~\textcolor{red}{This is not necessary, of course, but as of lately I'm leaning towards simple titles, so I let you decide}. Regarding the use of a novel salt--injection we believe this to be the first time this cell design has been used to study the formation of porous crystals in gels. \textcolor{red}{Could you please confirm and comment this?}.


\clearpage

\noindent
\begin{Large}
{\bf Response to Referee \#3}
\end{Large}


\vskip 0.5cm

We would like to thank the Referee for her/his comments. In the following we respond to the new points raised.

\begin{referee}
I was aware of the different frame rates on the two movies but I agree that the time stamps are helpful. Figure S5 addresses my concern about reproducibility at the single state point, and the different geometry itself is helpful too. I think the authors should be up front in the manuscript that the mechanism discussed is found in a very specific window (from the figures and SI, I think it is $\phi\sim 0.33$ and $c_p~0.39$ (plus minus) 0.01 mg/g) by moving this point into the abstract.
\end{referee}

In this new version of the manuscript we follow the Referee's recommendation and write explicitly in the abstract the condition for the formation of crystal-gels. We stress that the conditions for crystal-gel formation are not stringent and well within the experimental accessible states, as we previously demonstrated in our reproducibility tests inside a capillary.

\begin{referee}
The addition of the SI movie 3 and the discussion of Figure 2b addresses this concern for me. It's the percolation that is critical to the authors claim of a crystal gel - the percolated structure needs to persist over long times, otherwise, for example, the gel wouldn't have a finite low frequency elastic modulus. I think SI move 3 does show this.

3) The data quality of Figure 1c, Figure 1d appear insufficient to classify the cp = 0.38 mg/mg state point as different from the others. The authors should add replications, errors, and significance testing if they wish to distinguish the dimensionality (Fig 1c) and maximum volume fraction (Fig 1d) of the 0.38 mg/ml state point from the others.

I continue to think that the evidence for different behavior in the cp = 0.38 mg/g state point from Figure 1 c,d is weak. Supplemental Figure 3 is more convincing that the final state points are different. 
\end{referee}

We fully agree that the evidence from Fig.~1 c,d was not conclusive, but the goal of those figures was not to provide conclusive evidence, but to provide a comprehensive study of the early-stages of spinodal decomposition. In this context the Fig.~1 c,d only provide hints that the cp = 0.38 mg/g state point behaves differently from other state points. The rest of the manuscript and the Supplementary materials is devoted to providing the conclusive evidence that that is the case. We have made this more clear in the new manuscript.


\textcolor{red}{The Referee really likes Supplementary Fig3, so at this point if we want to answer him fully, we should probably replace Fig. 3c with Supplementary Figure 3. What do you think?}


\begin{referee}
(Yet, in the new section, the authors write: "This result clearly indicates that a single state point follows a different arrest mechanism, in which phase separation is arrested by crystallization and not by glassiness." Is that really true? It seems overstated. Don't Figure 1 c,d and SF3 just establish that the end point is different? They don't say anything about the mechanism, which would require kinetic analysis, as comes later. And it doesn't say anything about arrest, which would require dynamics.)
\end{referee}

In the new version of the manuscript we have removed that sentence, and replaced it with:


\emph{
These structural analysis provide some evidence that the state point $\phi\approx 0.33$ and $c_p=0.38\,$mg/g could be following a different arrest mechanism, in which phase separation is arrested by crystallization and not by glassiness. In the reminder of this manuscript we investigate this state more closely, to confirm these early results.}

We also agree that our previous manuscript didn't say much about the actual kinetic mechanism by which crystal-gels form. This point was also raised by Referee \#1 in noting that we didn't provide any additional insight in the phase separation process. In order to fully address this issue, we have implemented a kinetic analysis of the trajectories at all state points, computing the probability of bond-breaking between consecutive frames. This new kinetic analysis reveals that at low-polymer concentrations a new mechanism (stress-driven aging) is taking place, which has the effect of increasing the average coordination of the colloidal particles in the liquid phase.
We believe this to be a novel observation of this phenomena. Most past simulation works have neglected, or not fully taken into account, the role of hydrodynamic interactions (HI) in the phase separation process. As was shows for example in Ref.~\cite{furukawa2010key}, the phase separation process without HI produces rather compact structures, which would then easily crystallize to form compact nuclei. This is not what happens if one take HI into consideration. Due to the incompressibility condition, the liquid network has a lower-connectivity, and the viscoelastic phase separation results in the accumulation of mechanical stress on these liquid bonds. To things can happen depending on the polymer concentration. At high-polymer concentration, the bonds are not-easily broken, and the network which is first formed cannot undergo structural reorganisation. This inevitably leads to gelation. On the other side, if the polymer concentration is low enough, the liquid network can reorganise with a rather sharp transition to a more compact network. This condition is necessary for the appearance of the first crystalline nuclei. In our new manuscript, we are able to follow the kinetics of these processes by looking at the time evolution of the bond-breaking probability. The different network states are in fact characterised by different bond-breaking probabilities and by different connectives. In Fig.~2a,b we show these quantities for high-polymer concentration samples, where spinodal decomposition is the only process that occurs. At low-polymer concentrations (Fig.~2c,d) instead, the network undergoes a sharp transition to a state with lower bond-breaking probability and higher connectivity. This transition is the stress-driven aging process, and increases the connectivity of the network. Despite being theoretically predicted, Ref.~\cite{tanaka2000,tanaka2007spontaneous}, this transition was not directly observed before.

\begin{referee}
 The revised Figure 4, along with Figure S4, are very helpful to resolving the relative contributions of the three crystalline pathways. In Figure 4b,c, I still think that surface particles should somehow be distinguished - they represent a distinct state and a kind of consistency check, because the different crystallization pathways should all be mediated by this state. Without including the surface states in the primary figure, I don't see how the reader can inspect Figure S4, and get to the 54/34/11 abundances for direct, Bergeron and Ostwald processes. These abundances are the crucial claim of the paper.

In summary the confirmatory experiments (Fig S5 and Movie S3), the additional analysis of the crystalline state (Fig 3S) and the full transition analysis (rev Fig 4 and F S4) offer significant new support of the authors' claims; the authors have something new relative to prior literature. At this point, I think that the authors should: (i) better acknowledge that what is observed here is found in a quite specific region of the state diagram ($phi\sim 0.33$ and $c_p\sim 0.39$ {plus minus} 0.01 mg/g); (ii) better connect Figure 4 and Figure S4 by accounting for surface states in the former figure.
 
\end{referee}

\textcolor{red}{The Referee insists on having surface particles represented in Fig.6b. Even if I think that the figure becomes more complicated this way, maybe we should just update the figure. I'm sending Mathieu the dataset with the fraction of surface particles and liquid particles in case you agree on this change. If you do, I'll also update the description in the manuscript.}

In the new version of the manuscript we follow the Referee's recommendation and include in Fig. 6b the fraction of surface particles, and update the description accordingly.
% We remind that the 54/34/11 abundances was obtained by looking at the phase history of each colloidal particle. Fig. 6b and Supplementary Fig. 4 don't contain information about the phase history of individual particles, just about the total number of particles in each phase, and the time-dependence of the transition probabilities. Of course these quantities are closely connected, and we hope to have provided a detailed description of the process.

\clearpage

\bibliographystyle{naturemag2}
\bibliography{biblio}


\end{document}
